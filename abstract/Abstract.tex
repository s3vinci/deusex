\begin{abstract}
The fruit fly memorises previous experiences of different smells along
with information on whether the smell was associated with reward or
punishment. The location of thism memory is the mushroom body (MB),
an area comprised of thousands of neurons that has been implicated
in olfactory memory and decision-making. More precisely it is thought
to be held in the synapses unto a small group of so called \textquotedbl{}mushroom
body output neurons\textquotedbl{} (MBONs) controls approach and retreat
behaviour. They receive their input from other mushroom body neurons
and are also affected by dopamine. Experimental evidence suggesting
that dopamine controls both reward and aversive based learning (\citealp{Perisse:2013fp}).
It is thus believed that MBONs form a ``valence'' coding space;
depending on the combined firing rates of ``retreat'' and ``approach''
biasing MBONs, each odour will be considered to be either appetitive,
aversive or neutral by the fly (\citealp{Hige:2015er}). In this vein,
a recent study has shown that one such MBON decreases its activity
after appetitive training and increases its activity after aversive
learning (\citealt{Owald:2015cn}). 

To understand the mechanisms behind dopamine mediated learning of
olfactory valence, we have developped a spiking model of plasticity
in the olfactory circuit of the fruit fly. In our computational model
we can reproduce experimental evidence that show bi-directional change
of firing rate in MBONs. Furthermore we visualize the coding space
formed by the firing rates of MBONs and observe how odours change
their position in the valence coding space depending on the learning
paradigm used. Finally we make experimentally verifiable predictions,
proposing that if only the ``retreat'' biasing group of MBONs change
their firing rates bi-directionally as current evidence suggests,
aversive memories are more stable than appetitive memories. We also
predict that bi-directional change of firing rate can enhance the
discriminability between odours of different valences.

In the future, we propose to refine our olfactory valence learning
model of the fruit fly by incorporating recent anatomical, behavioural
and electrophysiological data available. With the help of state of
the art computational models and close experimental collaboration,
we hope to further our understanding of how reward and punishment
activated dopaminergic neurons guide learning in the fruit fly, as
well as general mechanisms that govern valence learning across different
species. \end{abstract}

